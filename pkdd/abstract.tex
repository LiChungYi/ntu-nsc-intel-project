\begin{abstract}

Datasets gathered from sensor networks often suffer from a significant fraction of missing data, due to issues such as 
communication interference, sensor interference, power depletion, and hardware failure. 
Many standard data analysis tools such as classification engines, time-sequence pattern analysis modules, and even statistical 
tools are ill-equipped to deal with missing values---hence, there is a vital need for highly-accurate techniques for imputing
missing readings prior to analysis.

In this paper, we present novel imputation methods that take a ``Recommendation Systems'' view of the problem: 
the sensors and their readings at each time step are viewed as products and user product ratings, with the goal of estimating the missing ratings.
Sensor readings differ from product ratings, however, in that the former exhibit high correlation in both time and space.
For example, readings in adjacent time steps are usually similar.
To incorporate this property, we modify the widely successful Matrix Factorization (MF) approach for recommendation systems to model inter-sensor and intra-sensor correlations and learn latent relationships among these dimensions.
We evaluate the approach using two environmental sensor network datasets, one indoor and one outdoor, and
two imputation scenarios, corresponding to intermittent readings and failed sensors.
%The results show that our Temporally-Regularized MF (TR-MF) approach provides significantly higher estimation accuracy than 
%both (i) state-of-the-art recommendation models and (ii) state-of-the-art sensor data imputation approaches 
%%such as the Applying K-nearest neighbor (AKE) model.
%such as AKE.
%The experiments also show that adding spatial coordinate information into TR-MF yields even better results
%in the failed sensors scenario, but {\em not} in the intermittent readings scenario.

Next, we consider sensor networks with multiple sensor types at each node.  We present two techniques for extending our model to account for possible correlations among sensor types (e.g., temperature and
humidity). Our results show that both techniques 
can significantly improve the accuracy.% over TR-MF, and each has its strengths, depending on the observed variance in the readings.

Finally, we study how the imputed values affect the result of data analysis.
We consider a popular data analysis task---building regression-based prediction models---and show that,
compared to prior approaches for imputation, using our model leads to a much higher quality prediction model.

\end{abstract}

%\category{H.4}{Information Systems Applications}{Miscellaneous}
%A category including the fourth, optional field follows...
%\category{D.2.8}{Software Engineering}{Metrics}[complexity measures, performance measures]

%\terms{Delphi theory}

%Omit for submission, to save space:
%\keywords{missing data recovery, matrix factorization, tensor factorization, temporal regularization, multivariate learning, data imputation}
