\section{Conclusion}  \label{sec:conc}
This paper proposes usage of factorization-based models for missing data estimation.
In contrast to many existing knowledge driven approaches which make stronger assumptions about the data (e.g., assume that nearby sensor nodes have higher similarity; or assume the so-called ``neighborhood area'' is at the same radius away from the center in every direction), our data-driven factorization model learns the inter-sensor and intra-sensor correlations through exploiting their latent similarity.
Furthermore, we show that the additional knowledge such as the spatial relationships among sensors can seamlessly be incorporated into our model through regularization terms (e.g., STR-MF) if needed.
Our experiments suggested that the temporal regularization is very helpful in general, while spatial information is useful only when the existing data are insufficient for the model to learn the inter-sensor relationships.
Finally, we propose an update time $\Theta(KN + KR)$ multivariate tensor factorization model whose training complexity does not grow with its order, which allows the users to add arbitrary more dimensions or features into the imputation model without significantly increasing the computation burden.
We believe that the factorization models will become a very important kind of missing data estimation technique for WSN in the near future, and we envision that our work can establish a foundation for more advanced research in this direction.
