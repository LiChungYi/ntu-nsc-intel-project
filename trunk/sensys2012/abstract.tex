\begin{abstract}
Data sets gathered from large-scale sensor networks are often prone to suffer from a significant fraction of missing data due to issues such as communication interference, power depletion, or hardware failure. 
Many standard data analysis tools such as classification engines, time-sequence pattern analysis modules, and even statistical tools are ill-equipped to deal will missing values, hence the need to impute missing readings prior to analysis.
We develop novel imputation methods which take a ``Recommendation Systems'' view of the problem: the sensors and their readings at each time step are viewed as users and product ratings with the goal of estimating the missing ratings.
Sensor readings differ from product ratings in that the former exhibit high correlation in both time and space.
To incorporate this property, we modify the widely successful Matrix Factorization (MF) approach for recommendation systems to model temporal correlations and learn latent relationships between sensors.
We evaluate the approach using two different environmental sensor network datasets, one indoor and one outdoor.
We show that a temporally-regularized MF significantly outperforms the state-of-the-art recommendation models as well as the state-of-the-art missing data estimation approaches such as the hybrid-KNN model.
Interestingly, incorporation of a spatial information into the formulation is shown to be ineffective as our collaborative-filtering based approach, after introducing a tensor-factorization model to further improve the imputation quality, is able to discover such latent correlations among sensors.
Furthermore, we extend the approach to possibly correlated signals of different sensor types (e.g.\ temperature and humidity) on the same sensor node.
Finally, experiments are conducted to demonstrate that a more accurate imputation method, such as we propose, can indeed lead to a higher-quality prediction model.
\end{abstract}

%\category{H.4}{Information Systems Applications}{Miscellaneous}
%A category including the fourth, optional field follows...
%\category{D.2.8}{Software Engineering}{Metrics}[complexity measures, performance measures]

%\terms{Delphi theory}

\keywords{missing data recovery, matrix factorization, tensor factorization, temporal regularization, multivariate learning, data imputation}
