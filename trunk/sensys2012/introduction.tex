\redtext{Note: Neither the abstract nor the introduction have been updated to reflect our recent findings that spatial
regularization can help for the temporal split case.}

\section{Introduction}

%\subsection{Problem Statement \redtext{(what is the problem to be solved?)}}
%We develop a centralized method for estimating missing data in sensor network datasets, a procedure which is crucial for subsequent analysis.
%Missing data imputation (a term from Statistics) has a long history, and while there are many existing algorithms which estimate missing sensor data (as documented in Section \ref{sec:rw}), few of these take advantage of the time and inter-sensor correlations inherent in the WSN datasets.
%Furthermore, distributed approaches~\cite{xiao2006space,nowak2003distributed} are often limited to providing estimations or decisions based on sensors in the immediate neighborhood and rely on the deployed sensors having adequate computational power to perform such calculations.
%Not suffering from these issues, the centralized approach we have taken enables a global solution (utilizing all observations available from the sensor network) and provides a vital backdrop for our novel highly-accurate sensor network data imputation technique.

%\subsection{Relevance (why is WSN dataset analysis an important topic?)}
%The increasing pervasiveness of Wireless Sensor Network (WSN) deployments is reflected by the recent coinage of terms such as ``Internet of Things'' and ``Machine-to-Machine'' to describe this growing revolution~\cite{ashton2009internet,gershenfeld2004internet,nokia2004machine,lawton2004machine}.
%Facilitated by a sharp reduction in hardware costs~\cite{estrin2000special}, this growing trend (beyond providing for the admission of new phrases into the vernacular) has led to an explosion in the amount and variety of sensor network todata in need of study.
%For this reason, analysis of WSN data has garnered much attention in recent years~\cite{balazinska2007data}.

%\subsection{Motivation \redtext{(why is data imputation needed for WSN datasets?)}}

% Data sets gathered from sensor networks often suffer from a significant fraction of missing data, due to issues such as 
% communication interference, sensor interference, power depletion, and hardware failure. 
% Many standard data analysis tools such as classification engines, time-sequence pattern analysis modules, and even statistical tools are ill-equipped to deal will missing values---hence, there is a need to impute missing readings prior to analysis.

Wireless sensor networks (WSNs) are especially susceptible to interference,
battery depletion, and other environmental and communications ailments
which lead to data loss.  Data sets gathered from sensor networks are
often missing 10\%--40\% of the possible readings, depending on the
severity of the environmental factors~\cite{give cites}.  These
missing values are problematic for data analysis tools such as
classification engines, time-sequence pattern analysis modules, and
other machine learning tasks, which are often ill-equipped to deal
with missing values.  Support Vector Machine
(SVM)~\cite{vapnik2000nature} and Multiple Regression (MR) analysis,
to name but a few examples, require complete datasets with no missing
values.  Popular statistical packages such as SAS, Stata, and R
provide a few default options for handling missing data, as a
preprocessing step, because the core algorithms need all data filled
in.  Typical options are (i) remove the entire row if there is a
missing value, or (ii) fill in the missing value (called {\em
imputation}) using either simple defaults like the average of
neighboring values or user-written code.  The first option discards
otherwise useful data, and in fact, may discard most of the rows in
datasets with high data loss.  Thus, imputation is a vital tool in the
preparation of sensor data for subsequent analysis. Because the
accuracy of the target data analysis depends on the accuracy of the
imputation, improvements in sensor data imputation can better serve
sensor network deployment objectives.

We consider the common setting where the goal is to collect all the sensor readings
in order to perform centralized analysis, while maximizing the lifetime of the WSN.
Sensor nodes are battery powered and may be energy-harvesting, and will often make
only a ``best effort'' attempt to transmit their readings back to the centralized 
collection point.  This contributes to the prevalence of missing data, furthering
the value of effective imputation techniques.

%\subsection{Background \redtext{(what solutions currently exist?)}}

\subsection{Existing Imputation Techniques}

Imputation techniques as applied to sensor data can be divided into three categories:
temporal methods (i.e., estimation using the observations from the target sensor at nearby time-steps), 
spatial methods (i.e., estimation using neighboring sensor node observations), 
and hybrid spatio-temporal methods, as shown in Table~\ref{tbl:methods}.

\begin{table*}
\centering
\caption{Salient Methods for Sensor Data Imputation}
\label{tbl:methods}
\begin{tabular}{|l|l|l|} \hline
Collection&Hot-Deck Imputation&Prediction Models\\ \hline
\multirow{2}{*} {Temporal} & Last-seen~\cite{Granger:lastseen} & Linear Interpolation\\ 
& Mean&\\ \hline
\multirow{3}{*}{Spatial}& WARM~\cite{le2005estimating}& DEPM~\cite{li2008data}\\ 
&FARM~\cite{Gruenwald:FARM}&K-NN~\cite{pan2010k}\\ 
&&Multi-Im~\cite{yuan2000multiple}\\\hline
\multirow{3}{*}{Hybrid Spatio-Temporal}& &DESM~\cite{li2008data}\\ 
&STI~\cite{Jian-Zhong:STI}&AKE~\cite{pan2010k}\\
&&Imputation Method~\cite{Lim:robust} \\\hline \end{tabular}
\end{table*}

%The feasibility of estimating missing sensor observations based on historical data is grounded by the known temporal correlation in WSN data~\cite{akyildiz2004exploiting}.
%Moreover, where there is a potential for global communication issues to affect the availability for sensor node observations \emph{en masse} during a given duration of time, utilizing spatial correlations as a basis for estimation may be not be possible.

{\em Temporal methods} leverage the temporal correlation among
readings by the same sensor node; salient methods include observed
data mean~\cite{madden2005tinydb,setz2009combining}, last
seen~\cite{Granger:lastseen}, and linear interpolation.  These methods
suffer, however, when there are long temporal gaps of data for a given
sensor; such gaps can be frequent in WSNs due to power depletion in
energy-harvesting sensors, communication ailments that last multiple
time steps, etc.  As a result, the usefulness of temporal imputation
methods drop rapidly as the number of consecutively missing readings
becomes large.

% (i.e., as can happen when intermittent communications starvation occurs in large WSNs).  

\begin{figure}[ht]
\centering
\includegraphics[scale=0.35]{house_floorplan.png}
\caption{Example home floorplan showing five deployed temperature sensors} \label{house_floorplan}
\label{fig:example_home_floorplan}
\end{figure}

{\em Spatial methods} leverage the spatial correlation among readings
by nearby sensor nodes; salient methods include associations rule
mining (e.g., WARM~\cite{le2005estimating} and
FARM~\cite{Gruenwald:FARM}) 
%\cite{jiang2007estimating} 
and weighted functions of nearby sensors (e.g., DEPM~\cite{li2008data},
K-NN\cite{pan2010k}, and Multi-Im\cite{yuan2000multiple}).
These methods require reasonably accurate spatial coordinates, and
more importantly, suffer when there are ailments that affect entire
spatial regions such as may arise in the presence of a large obstacle
to sensing and/or communication.
Spatial methods also fail to take into account barriers or other
sources of sharp environmental gradients that may deter the usage of
spatial information as a first-order inter-sensor node correlation
approximation.  For example, in Figure~\ref{house_floorplan}, we find
sensors $1$ \& $2$ deployed in the kitchen, nearby the stove and
outside window, respectively. While these sensors are close in
proximity, assuming the stove is in use and the temperature outside is
cold, there may be a large temperature difference between these two
sensors despite their close proximity.  Similarly, sensors $2$ \& $5$
may be quite uncorrelated despite their relative proximity due to the
wall between them and the presence of kitchen or laundry appliance
use.  On the other hand, sensors $3$ \& $4$, while located further
from one another may be quite correlated despite their remote
placement as they are both near an outside wall and both within the
same room.  The calculation of
inter-node signal strength or line of sight distance between nodes can
help to mitigate the issues of spatially-based imputation, though it is not
a complete remedy.  In the end, the incorporation of spatial
information can lead to worse imputation results as non-existent (or
at least inconsistent) correlations are imposed between sensors.

Certain methods consider not strictly the distance between sensors,
but instead establish a ``neighborhood of influence'' whose size
becomes a tuning parameter of this approach, which adds to the
complexity of this approach.

Finally, {\em hybrid spatio-temporal methods} consider both the
temporal and spatial correlation; salient methods include
STI~\cite{Jian-Zhong:STI}, DESM~\cite{li2008data}, 
AKE~\cite{pan2010k}, and Imputation Method~\cite{Lim:robust}.  These
have the potential advantage of using both types of correlation in
imputation, but can suffer from the spatial correlation issues
discussed above.

%Hybrid methods of temporal and spatial approaches are less common in the literature.
%For example, the average of the temporal approach of linear interpolation and the spatial approach of multivariate regression has been reported[8].
%Strictly speaking, this approach can be thought of as an ensemble approach between the two methods rather than a fully-integrated approach which considers both temporal and spatial aspects of WSN data.

%\subsection{Research Gap Identification (why are current approaches inadequate?)}
%Accurate imputation of missing sensor network observations is crucial to allow for effective subsequent analysis.
%While there are many existing algorithms which estimate missing data (as documented in the following Related Works section), few of these take advantage of the time and space dependencies inherent in the WSN datasets during the data imputation process.
%As a result, the accuracy of such approaches is limited.

%\subsection{Method Overview \redtext{(what is our approach to bridge the research gap?)}}

\subsection{Our Approach: Collaborative Filtering}

In this paper, we employ a novel collaborative-filtering (CF) approach
to sensor data imputation inspired by the field of Recommendation
Systems.  In typical CF approaches, the elements of interest are users
and items (e.g., products), and the values are user ratings of those
items (as in the left-hand side of
Figure~\ref{recommend_imputation}).  Typically, most of the ratings
are missing, and the goal is to predict (impute) the missing ratings
in order to ``recommend'' items to users.  By viewing sensors as
items, users as time steps, and readings as ratings (as illustrated in
Figure~\ref{recommend_imputation}), we can apply CF
techniques to perform sensor data imputation. (Alternatively, sensors
can be viewed as users and items as time steps---the mapping is
irrelevant to the CF formulation.)  In particular, we focus on the
widely successful {\em Matrix Factorization} (MF) technique for CF.

\begin{figure}[H]
\centering
\includegraphics[scale=0.35]{recommend_imputation.png}
\caption{Bridge from Recommendation Systems to Sensor Data Imputation} 
\label{recommend_imputation}
\end{figure}

Sensor readings differ from user ratings, however, in that the former
exhibit high correlation in both time and space (subject to the above
caveats on space).  To incorporate this property, we first modify MF
to model temporal correlations and learn latent relationships among
sensors.  Specifically, we add {\em temporal regulation} bias terms to
MF---we call this {\em temporally-regulated MF} (TR-MF). Similarily, we can enforce spatial correlation among sensors into our model by adding the {\em spatial regulation} bias terms, which we call the {\em spatially-temporally-regulated} MF (STR-MF).

Second, we consider sensor networks with multiple sensor types at each node.
We are able to exploit such heterogeneous sensor information in our
solution, which few other methods have proposed a proper way to
incorporate.  Our method incorporates these heterogeneous sensor
signals (for example, estimating the temperature at a given sensor
node utilizing the humidity, temperature, and light trends from other
sensors in the network) to provide more accurate imputation than 
prior approaches.  We present two techniques for extending 
TR-MF to account for possible correlations among sensor types: {\em Multivariate TR-MF} and 
{\em Temporally-regularized Tensor Factorization} (TRTF).

We evaluate our approaches using two environmental sensor network
datasets, one indoor and one outdoor.
Both datasets record temperature, humidity, and light within its deployed
environment.  Each dataset has an initial missing rate (strengthening
the claim that missing data in WSNs is a common issue), to which we
additionally cover known observations to use for validation and
testing purposes.  We study two patterns for missing data: (i) covering
random readings (modeling intermittent reading failures) and (ii)
covering all readings for a subset of sensor nodes from a particular time onwards 
(modeling long temporal gaps such as with failed sensors).

Our results show that TR-MF provides significantly higher estimation accuracy than 
both (i) state-of-the-art recommendation models and (ii) state-of-the-art sensor data imputation approaches 
such as AKE which has the top performance among hybrid spatio-temporal methods.
Furthermore, adding spatial information into TR-MF is shown to be useful only when there are less temporal information provided through data. ---with sufficient readings, TR-MF can capture the latent relationships among sensors, including spatial correlations.  

For the heterogeneous setting, our results show that both TRTF and
Multivariate TR-MF can be more accurate than prior approaches under certain circumstances. Finally, we consider a popular data analysis task---building regression-based prediction models---and show that, compared to prior approaches, applying our more-accurate imputation techniques leads to higher-quality prediction models.

These results validate our novel approach of equating sensor imputation with recommendation systems.  CF approaches such as Matrix Factorization and Tensor Factorization are adept at handling sceanarios with large numbers of missing values.  Correlations are captured by grouping correlated sensor nodes and correlated time
steps---unlike prior sensor data imputation approaches, our CF
approaches use this {\em latent} information to impute values, and optimize the evaluation metrics directly. 
Moreover, our CF approaches are global, taking into account all
collected observations, and not overly tied to distance-based spatial
correlations.  For example, they can capture the correlations between
distant sensors $3$ \& $4$ in Figure~\ref{house_floorplan}, while
grouping sensors $1$, $2$ \& $5$ only if the observed readings warrant
it.  The CF framework also provides a unified approach to incorporate
any number of additional sensor types for even more accurate imputation.



%The reason we believe our method outperforms the results of other methods are as follows.
%\begin{itemize}
%\item A CF approach utilizes {\em latent} information between sensors, e.g., inter-sensor correlation
%\item utilizing heterogeneous sensor information (e.g.\ utilizing humidity information when estimating temperature) provides additional features which enables a more refined estimation model
%\item we provide an efficient optimization method to learn the inherent model parameters effectively
%\item our method provides a global solution, where all sensor observations available in the dataset can potentially aid in the estimation of a given missing observation
%\end{itemize}



% Given these experimental conditions, we show that our temporal and spatial-oriented collaborative filtering approach to data imputation for WSNs performs more accurately than existing methods such as linear regression and hybrid-kNN.
% In Matrix Factorization, we impose the temporal regularization and Temporal-Regularied MF show the best performance compared to all competitor algorithmss.
% On top of that, we add in spatial regularization and multivariate learning, and we discuss under which circumstance these two are applicable and how they can even improve our models.
% We propose the Tensor Factorization model for missing data recovery. The TF model use additional dimension to capture the correlations of features such as temporal, spatial or heterogeneous signal correlations. The conventional Tensor decomposition technology can only apply on dense tensor. The study of Tensor Factorization model in Collaborative Filtering for recommendation recently, but not practical in sensor network. We  have adapted the TF model for missing value estimation. 

\subsection{Contributions}

In summary, the main contributions of this paper are:
\begin{itemize}
\item We propose viewing sensor data imputation as a recommendation systems problem, and apply state-of-the-art collaborative filtering methods of recommendation systems (namely, matrix and tensor factorization) to the sensor network domain.
\item We augment collaborative filtering with temporal regularization and multi-sensor signals, and provide efficient optimization methods to learn the inherent model parameters effectively.
\item We present an empirical study on two sensor datasets, considering two
imputation tasks that correspond to intermittent readings and failed
sensors. The results show that our proposed approaches provide
significantly higher estimation accuracy than state-of-the-art prior
approaches, and moreover, such accuracy improvements can result in the
generation of higher-quality prediction models.
\end{itemize}

% which finds that our method works well without the incorporation spatial information and that our approach can facilitate the generation of better prediction models

\subsection{Paper Organization}

The remainder of our paper is organized as follows.  Related Work is
reviewed in Section \ref{sec:rw}.  In Sections \ref{sec:mf} and
\ref{sec:tf} we describe our Matrix and Tensor Factorization
approaches to WSN data imputation, respectively. Section \ref{sec:exp} discusses the experiment results. Sections \ref{sec:disc} and \ref{sec:conc} provide discussion of our findings
and the conclusion.
