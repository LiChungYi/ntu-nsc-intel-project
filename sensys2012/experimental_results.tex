\subsection{Experimental Results on Multivariate Learning}
%\subsubsection{Multivariate TR-MF}
%weilun please double check this section

%more?
To evaluate multivariate imputation models, in the experiments we allow temporal and humidity information to mutually enhance the predictions of the other. Here we compare the two proposed multivariate imputation models: MTR-MF and TF to with the univariate model TF-MF, which has been demonstrated to outperform other models in the previous section. 

There are two different scenarios we are concerned in this experiment. For illustration purpose, let's assume we are predicting the missing entries in the temperature matrix. In the first scenario (denoted as TR-MF-all), we assume the humidity sensors do not miss any readings. The goal here is to evaluate whether using all the humidity information allows our model to improve the prediction in temparature. In the second scenario (denoted as TR-MF-part), we assume the humidity readings are missing altogether with the temparature readings. Note that such cases happen quite often in WSN due to communication loss or sensor malfunction. In TR-MF-part, MTR-MF can be performed as in TR-MF-all. However, to predict an entry in temperature matrix, TF model does require to use the correspending entry in the humidity matrix. Such issue can be addressed by first predicting the missing humidity readings based on the humidity TR-MF model, and then apply TF.

The results are shown in Figure 10. Let's first focus on the TF model. First of all, we observe that TF-part and TF-all produce similar quality of results. The results further show that TF model outperforms all other models significantly in traffic dataset, but does not do as good in Berkeley dataset. The main reason we believe is as the follows. To perform TF, it is required that each dimension of the tensor be discretized. While the sensor and time-step dimensions are already discretized, the third dimension (i.e. humidity or temparature) is not. Therefore we need to perform discritization first before conducting tensor factorization. The problem for the Berkeley dataset is that as an indoor dataset, the variances of its temparature and humidity readings are not as large as those of the traffic dataset, therefore a uniform discritization would lead to uneven split of values on this 3rd-dimension, which then limits its power in explaining the data.  
As to the MTR-MF model, the results show that MTR-MF-part does not outperform TR-MF but MTR-MF-all does (except in consecutive missing Berkeley dataset). That says, the MTR-MF model is not helpful when multiple sensors are mssing altogether, but is likely to be useful when the heterogeneous signal is fully available. 

%\redtext{don't use result here!!! see my google doc!}
%\begin{table}[htbp]
%\setlength{\tabcolsep}{2pt}
%\centering
%\caption{Multivariate RMSE (Berkeley, random)}
%\label{table:multi_berkeley_random}
%\begin{tabular}{r | r r r r r}
%train	&TR-MF	&MtMF-Train	&MtMF-all &TF-Train & TF-all \\ \hline
%humid10\%	&0.1424	&0.1420	&0.1222 &0.510&0.1392\\
%humid20\%	&0.1135	&0.1135	&0.0973&0.1412&0.1243\\
%humid40\%	&0.0916	&0.0909	&0.0822&0.1113&0.0944\\
%humid60\%	&0.0817	&0.0806	&0.0758&0.1086&0.0842\\
%humid80\%	&0.0757	&0.0748	&0.0709&0.0918&0.0781\\
%humid85\%	&0.0751	&0.0739	&0.0712&0.0812&0.0772\\ \hline
% temp10\%	&0.1137	&0.1148	&0.0729&0.0878&0.0701\\
% temp20\%	&0.0462	&0.0481	&0.0369&0.0501&0.0361\\
% temp40\%	&0.0316	&0.0328	&0.0263&0.0303&0.0280\\
% temp60\%	&0.023	&0.0232	&0.0201&0.0243&0.0234\\
% temp80\%	&0.0182	&0.0183	&0.0166&0.0193&0.0186\\
% temp85\%	&0.0153	&0.0154	&0.0143&0.0175&0.0171\\
%\end{tabular}
%\end{table}
%
%\begin{table}[htbp]
%\setlength{\tabcolsep}{2pt}
%\centering
%\caption{Multivariate RMSE (Berkeley, temporal)}
%\label{table:multi_berkeley_temporal}
%\begin{tabular}{r | r r r r r}
%train	&TR-MF	&MtMF-Train	&MtMF-all &TF-Train&TF-all \\ \hline
%humid10t	&0.957&0.996& 	0.991&1.110&1.291\\
%humid20t	&0.796&0.852& 	0.846&1.127&1.009\\
%humid40t	&0.771&0.835& 	0.807&0.813&0.806\\
%humid60t	&0.540&0.887& 	0.880&0.896&0.841\\
%humid80t	&0.447&0.483& 	0.480&0.461&0.406\\
%humid85t	&0.323&0.356& 	0.348&0.332&0.321\\	\hline
% temp10t	&0.515&0.567& 	0.555&1.525&1.109\\
% temp20t	&0.392&0.491& 	0.485&0.705&0.512\\
% temp40t	&0.310&0.380& 	0.347&0.356&0.338\\
% temp60t	&0.206&0.309& 	0.257&0.307&0.253\\
% temp80t	&0.132&0.429& 	0.432&0.243&0.215\\
% temp85t	&0.088&0.122& 	0.114&0.121&0.099\\
%\end{tabular}
%\end{table}
%
%\begin{table}[htbp]
%\setlength{\tabcolsep}{2pt}
%\centering
%\caption{Multivariate RMSE (traffic Data, Random)}
%\label{table_multi_traffic_random}
%\begin{tabular}{r | r r r r r}
%train	&TR-MF	&MtMF-Train	&MtMF-all &TF-train & TF-all\\ \hline
%humid10\%	&3.524 	&3.486 	&2.291&2.190&2.209\\  
%humid20\%	&2.583 	&2.558 	&1.796&1.461&1.821\\
%humid40\%	&1.932 	&1.921 	&1.523&1.409&1.609\\
%humid60\%	&1.664 	&1.649 	&1.408&1.357&1.472\\
%humid80\%	&1.565 	&1.546 	&1.366&1.292&1.241\\
%humid85\%	&1.503 	&1.489 	&1.326&1.291&1.289\\ \hline
% temp10\%	&1.214 	&1.201 	&0.722&0.641&0.727\\
% temp20\%	&0.898 	&0.881 	&0.597&0.531&0.510\\
% temp40\%	&0.689 	&0.676 	&0.519&0.429&0.419\\
% temp60\%	&0.585 	&0.579 	&0.478&0.389&0.411\\
% temp80\%	&0.551 	&0.544 	&0.466&0.401&0.410\\
% temp85\%	&0.520 	&0.510 	&0.434&0.372&0.396\\
%\end{tabular}
%\end{table}
%
%\begin{table}[htbp]
%\setlength{\tabcolsep}{2pt}
%\centering
%\caption{Multivariate RMSE (traffic Data, temporal)}
%\label{table_multi_traffic_temporal}
%\begin{tabular}{r | r r r r r}
%train	&TR-MF	&MtMF-Train	&MtMF-all &TF-Train &TF-all\\ \hline
%humid10\%	&5.195 	&5.346 	&4.103&3.391&3.389\\  
%humid20\%	&5.487 	&5.565 	&4.263&3.571&3.399\\
%humid40\%	&5.782 	&6.009 	&3.968&3.544&3.537\\
%humid60\%	&4.954 	&4.918 	&3.877&3.512&3.694\\
%humid80\%	&4.564 	&4.700 	&3.346&2.881&2.862\\
%humid85\%	&4.248 	&4.248 	&3.254&2.329&2.332\\ \hline
% temp10\%	&1.700 	&1.708 	&1.194&1.251&1.173\\
% temp20\%	&1.812 	&1.815 	&1.144&1.209&1.307\\
% temp40\%	&1.832 	&1.835 	&1.033&1.288&1.293\\
% temp60\%	&1.666 	&1.646 	&1.220&1.239&1.201\\
% temp80\%	&1.430 	&1.437 	&0.916&0.911&0.929\\
% temp85\%	&1.441 	&1.345 	&0.915&0.912&0.912\\
%\end{tabular}
%\end{table}
%\subsubsection{Tensor Factorization} % need to be combined with multi TR-MF 

%We compare our TF method with models not incorporating heterogeneous sensor correlations in this section.
%Table ? and ? show the TF result of random split and temporal split on Berkeley dataset while table ? and ? show the TF result of random split and temporal split on traffic dataset.

\subsection{Designing Prediction Models after Imputation}\label{subsec:furtherPredict}
As stated previously, one main reason to conduct data imputation is that most off-the-shelf data analysis tools cannot deal with inputs with missing values. Here one practical question to ask is: whether our imputation outcomes can indeed lead to the development of a better analysis model? Here we conduct an experiment using the 20 sensors in the traffic dataset to build two regression models (linear regression and support-vector regression). In the epxeriment, we first remove the gateway signal from the matrix, and use TR-MF as well as the other competitor models (including global mean, LI, AKE) to fill in all the missing values in the remaining 20 sensors. Then for each imputation model, we use the filled-in readings of these 20 sensors as the inputs X and the gateway values as the output Y to train the regression models for gateway-value prediction. We divide the data randomly into 80\% training and 20\% testing, and show the results of both regression models in Table 5. The results confirm our hypothesis that the a better imputation model does lead to the production of a better prediction model, and TR-MF again outperforms the other models in this aspect.




\begin{table} [htbp]
\caption{RMSE of gateway prediction (use humidity as features to predict temperature) } \label{table:gateway prediction}
\setlength{\tabcolsep}{2pt}
\centering
\small
\begin{tabular} {c | r r r r | r r r r}
& \multicolumn{4}{ c|}{LR} & \multicolumn{4}{|c}{SVR}  \\ \hline
train & \begin{turn}{65}Mean\end{turn} & \begin{turn}{65}LI\end{turn} & \begin{turn}{65}AKE\end{turn}& \begin{turn}{65}TR-MF\end{turn}& \begin{turn}{65}Mean\end{turn} & \begin{turn}{65}LI\end{turn} & \begin{turn}{65}AKE\end{turn}& \begin{turn}{65}TR-MF\end{turn}  \\ \hline
    80\%   &6.021&4.973&3.812&3.001   &6.222&5.012&4.101&3.144\\
   60\%   &5.086&4.385&3.722&2.943    &5.524&4.551&3.972&2.807\\
    50\%    &4.992&4.180&3.158&2.964     &5.281&4.241&3.244&2.877\\
      30\%    &4.978&4.075&2.822&2.570     &5.003&4.175&3.058&2.692\\
      10\%    &4.642&3.868& 2.874&2.526 &4.502&3.713& 2.891&2.681\\ 
   5\%   &4.497&3.799&2.796&2.512      &4.323&4.018&2.895&2.409\\
\end{tabular}
\end{table}