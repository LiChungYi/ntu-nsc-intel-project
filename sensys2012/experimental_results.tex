\subsection{Basic Experimental Results} %from Chung-Yi
We compare linear interpolation~(LI), TRMF, AKE, DESM and STI on the two data sets. Note that the sensor location information is not available in traffic data set, so we assume the distance between all sensor node pairs are equal for DESM and STI.

%validation setting
\subsubsection{Random Split}
Table \ref{table:berkeley_random_hum}, \ref{table:berkeley_random_light} and \ref{table:berkeley_random_tem} show the result of random split on Berkeley data set. TRMF outperforms others greatly in all cases. On the other hand, linear interpolation is quite competitive, especially when the size of training set is large. This is because the sampling rate of Berkeley data set is fairly high and the temporal correlation is prominent. 

\begin{table}[htbp]
\centering
\caption{RMSE of (Berkeley, random, humidity)}
\label{table:berkeley_random_hum}
\begin{tabular}{ r | r r r r r}
	train	&LI	&TRMF	&AKE	&DESM	&STI\\ \hline
%	5\% & $ 0.319_{(2)} $ & $ \mathbf{ 0.198_{(1)} } $ & $ 0.354_{(3)} $ & $ 1.957_{(5)} $ & $ 1.220_{(4)} $ \\
	10\% & $ 0.171_{(2)} $ & $ \mathbf{ 0.142_{(1)} } $ & $ 0.264_{(3)} $ & $ 0.544_{(4)} $ & $ 1.220_{(5)} $ \\
	20\% & $ 0.127_{(2)} $ & $ \mathbf{ 0.114_{(1)} } $ & $ 0.244_{(3)} $ & $ 0.318_{(4)} $ & $ 1.472_{(5)} $ \\
	40\% & $ 0.095_{(2)} $ & $ \mathbf{ 0.092_{(1)} } $ & $ 0.199_{(3)} $ & $ 0.219_{(4)} $ & $ 1.648_{(5)} $ \\
	60\% & $ 0.085_{(2)} $ & $ \mathbf{ 0.082_{(1)} } $ & $ 0.181_{(4)} $ & $ 0.180_{(3)} $ & $ 1.649_{(5)} $ \\
	80\% & $ \mathbf{ 0.075_{(1)} } $ & $ 0.076_{(2)} $ & $ 0.127_{(3)} $ & $ 0.160_{(4)} $ & $ 1.604_{(5)} $ \\
	85\% & $ \mathbf{ 0.074_{(1)} } $ & $ 0.075_{(2)} $ & $ 0.121_{(3)} $ & $ 0.148_{(4)} $ & $ 1.549_{(5)} $ \\ \hline
	rank &1.71 &1.29 &3.14 &4.00 &4.86 \\
\end{tabular}
\end{table}

\begin{table} [htbp]
\centering
\caption{RMSE of (Berkeley, random, light)}
\label{table:berkeley_random_light}
\begin{tabular}{ r |  r r r r r}
	train	&LI	&TRMF	&AKE	&DESM	&STI\\ \hline
%	5\% & $ 70.8_{(2)} $ & $ \mathbf{ 51.4_{(1)} } $ & $ 74.8_{(3)} $ & $ 18531.4_{(5)} $ & $ 225.6_{(4)} $ \\
	10\% & $ 52.9_{(2)} $ & $ \mathbf{ 35.5_{(1)} } $ & $ 61.2_{(3)} $ & $ 6250.4_{(5)} $ & $ 258.2_{(4)} $ \\
	20\% & $ 38.7_{(2)} $ & $ \mathbf{ 28.2_{(1)} } $ & $ 53.0_{(3)} $ & $ 215.9_{(4)} $ & $ 311.0_{(5)} $ \\
	40\% & $ 29.9_{(2)} $ & $ \mathbf{ 21.2_{(1)} } $ & $ 41.4_{(3)} $ & $ 356.7_{(5)} $ & $ 356.3_{(4)} $ \\
	60\% & $ 25.0_{(2)} $ & $ \mathbf{ 17.2_{(1)} } $ & $ 33.7_{(3)} $ & $ 112.6_{(4)} $ & $ 366.9_{(5)} $ \\
	80\% & $ 23.9_{(2)} $ & $ \mathbf{ 17.7_{(1)} } $ & $ 27.9_{(3)} $ & $ 41.7_{(4)} $ & $ 363.3_{(5)} $ \\
	85\% & $ 20.9_{(2)} $ & $ \mathbf{ 14.4_{(1)} } $ & $ 24.3_{(3)} $ & $ 44.4_{(4)} $ & $ 354.2_{(5)} $ \\ \hline
	rank &2.00 &1.00 &3.00 &4.43 &4.57 \\
\end{tabular}
\end{table}

\begin{table}[htbp]
\centering
\caption{RMSE of (Berkeley, random, temperature)}
\label{table:berkeley_random_tem}
\begin{tabular}{ r | r r r r r}
	train	&LI	&TRMF	&AKE	&DESM	&STI\\ \hline
%	5\% & $ 15.725_{(4)} $ & $ 4.690_{(2)} $ & $ \mathbf{ 4.671_{(1)} } $ & $ 21.981_{(5)} $ & $ 14.998_{(3)} $ \\
	10\% & $ 11.630_{(4)} $ & $ 3.524_{(2)} $ & $ \mathbf{ 3.486_{(1)} } $ & $ 19.817_{(5)} $ & $ 10.625_{(3)} $ \\
	20\% & $ 7.269_{(4)} $ & $ 2.583_{(2)} $ & $ \mathbf{ 2.558_{(1)} } $ & $ 14.634_{(5)} $ & $ 6.745_{(3)} $ \\
	40\% & $ 4.233_{(3)} $ & $ 1.932_{(2)} $ & $ \mathbf{ 1.921_{(1)} } $ & $ 8.986_{(5)} $ & $ 5.021_{(4)} $ \\
	60\% & $ 3.184_{(3)} $ & $ 1.664_{(2)} $ & $ \mathbf{ 1.649_{(1)} } $ & $ 6.396_{(5)} $ & $ 4.710_{(4)} $ \\
	80\% & $ 2.690_{(3)} $ & $ 1.565_{(2)} $ & $ \mathbf{ 1.546_{(1)} } $ & $ 4.714_{(5)} $ & $ 4.605_{(4)} $ \\
	85\% & $ 2.588_{(3)} $ & $ 1.503_{(2)} $ & $ \mathbf{ 1.489_{(1)} } $ & $ 4.382_{(4)} $ & $ 4.671_{(5)} $ \\ \hline
	rank &3.43 &2.00 &1.00 &4.86 &3.71 \\
\end{tabular}
\end{table}

Table \ref{table:traffic_random_hum} and \ref{table:traffic_random_tem} show the result of random split on traffic data set. Again, TRMF outperforms other models significantly. However, the performance of linear interpolation deteriorates a lot. This corresponds to the low sampling rate in traffic data set. 

\begin{table} [htbp]
\centering
\caption{RMSE of (traffic, random, humidity)}
\label{table:traffic_random_hum}
\begin{tabular}{ r | r r r r r}
	train	&LI	&TRMF	&AKE	&DESM	&STI\\ \hline
%	5\% & $ 15.725_{(4)} $ & $ \mathbf{ 4.690_{(1)} } $ & $ 11.894_{(2)} $ & $ 21.981_{(5)} $ & $ 14.998_{(3)} $ \\
	10\% & $ 11.630_{(4)} $ & $ \mathbf{ 3.524_{(1)} } $ & $ 7.307_{(2)} $ & $ 19.817_{(5)} $ & $ 10.625_{(3)} $ \\
	20\% & $ 7.269_{(4)} $ & $ \mathbf{ 2.583_{(1)} } $ & $ 4.559_{(2)} $ & $ 14.634_{(5)} $ & $ 6.745_{(3)} $ \\
	40\% & $ 4.233_{(3)} $ & $ \mathbf{ 1.932_{(1)} } $ & $ 3.458_{(2)} $ & $ 8.986_{(5)} $ & $ 5.021_{(4)} $ \\
	60\% & $ 3.184_{(3)} $ & $ \mathbf{ 1.664_{(1)} } $ & $ 2.901_{(2)} $ & $ 6.396_{(5)} $ & $ 4.710_{(4)} $ \\
	80\% & $ 2.690_{(3)} $ & $ \mathbf{ 1.565_{(1)} } $ & $ 2.511_{(2)} $ & $ 4.714_{(5)} $ & $ 4.605_{(4)} $ \\
	85\% & $ 2.588_{(3)} $ & $ \mathbf{ 1.503_{(1)} } $ & $ 2.401_{(2)} $ & $ 4.382_{(4)} $ & $ 4.671_{(5)} $ \\ \hline
	rank &3.43 &1.00 &2.00 &4.86 &3.71 \\
\end{tabular}
\end{table}

\begin{table} [htbp]
\centering
\caption{RMSE of (traffic, random, temperature)}
\label{table:traffic_random_tem}
\begin{tabular}{ r | r r r r r}
	train	&LI	&TRMF	&AKE	&DESM	&STI\\ \hline
%	5\% & $ 5.306_{(4)} $ & $ \mathbf{ 1.632_{(1)} } $ & $ 4.014_{(2)} $ & $ 6.855_{(5)} $ & $ 5.149_{(3)} $ \\
	10\% & $ 4.000_{(4)} $ & $ \mathbf{ 1.214_{(1)} } $ & $ 2.509_{(2)} $ & $ 6.212_{(5)} $ & $ 3.662_{(3)} $ \\
	20\% & $ 2.508_{(4)} $ & $ \mathbf{ 0.898_{(1)} } $ & $ 1.538_{(2)} $ & $ 4.867_{(5)} $ & $ 2.261_{(3)} $ \\
	40\% & $ 1.477_{(3)} $ & $ \mathbf{ 0.689_{(1)} } $ & $ 1.192_{(2)} $ & $ 3.149_{(5)} $ & $ 1.732_{(4)} $ \\
	60\% & $ 1.101_{(3)} $ & $ \mathbf{ 0.585_{(1)} } $ & $ 1.005_{(2)} $ & $ 2.275_{(5)} $ & $ 1.597_{(4)} $ \\
	80\% & $ 0.938_{(3)} $ & $ \mathbf{ 0.551_{(1)} } $ & $ 0.885_{(2)} $ & $ 1.702_{(5)} $ & $ 1.574_{(4)} $ \\
	85\% & $ 0.915_{(3)} $ & $ \mathbf{ 0.519_{(1)} } $ & $ 0.866_{(2)} $ & $ 1.494_{(4)} $ & $ 1.585_{(5)} $ \\ \hline
	rank &3.43 &1.00 &2.00 &4.86 &3.71 \\
\end{tabular}
\end{table}

\subsubsection{Temporal Split}
\textbf{Table \ref{table:berkeley_temporal_hum}, \ref{table:berkeley_temporal_light} and \ref{table:berkeley_temporal_tem}} show the result of random split on Berkeley data set and \textbf{Table \ref{table:traffic_temporal_hum} and \ref{table:traffic_temporal_tem}} show the result of random split on traffic data set.
In the temporal split, it is clear that linear interpolation performs terribly since its prediction is simply the last reading of the sensor.

\begin{table}[htbp]
\centering
\caption{RMSE of (Berkeley, temporal, humidity)}
\label{table:berkeley_temporal_hum}
\begin{tabular}{ r | r r r r r}
	train	&LI	&TRMF	&AKE	&DESM	&STI\\ \hline
	85\%	&3.9669	&1.1250	&1.4492	&3.9682	&0.7876\\ 
	80\%	&4.1829	&1.0069	&1.6694	&4.1853	&0.7918\\
	70\%	&5.4211	&1.2549	&0.9690	&5.4217	&0.8651\\
	50\%	&6.4429	&0.8844	&1.0252	&6.4431	&0.9612\\
	30\%	&2.2225	&0.9314	&0.9280	&2.2335	&0.8643\\
	10\%	&1.2076	&0.5392	&0.6358	&1.2165	&0.6878\\
	 5\%	&2.3060	&0.4285	&0.7768	&2.3077	&0.7960
\end{tabular}
\end{table}

\begin{table}[htbp]
\centering
\caption{RMSE of (Berkeley, temporal, light)}
\label{table:berkeley_temporal_light}
\begin{tabular}{ r | r r r r r}
	train	&LI	&TRMF	&AKE	&DESM	&STI\\ \hline
	85\%	&330.55	&176.51	&196.39	&330.85	&251.32\\ 
	80\%	&320.02	&166.79	&239.09	&320.00	&251.35\\
	70\%	&497.84	&120.53	&257.11	&499.70	&206.58\\
	50\%	&194.54	&58.39	&68.55	&195.17	&208.28\\
	30\%	&312.10	&33.83	&77.45	&312.12	&289.20\\
	10\%	&293.47	&20.60	&84.89	&293.87	&213.90\\
	 5\%	&277.85	&7.81	&79.05	&280.36	& 92.72
\end{tabular}
\end{table}


\begin{table}[htbp]
\centering
\caption{RMSE of (Berkeley, temporal, temperature)}
\label{table:berkeley_temporal_tem}
\begin{tabular}{ r | r r r r r}
	train	&LI	&TRMF	&AKE	&DESM	&STI\\ \hline
	85\%	&3.0873	&0.6098	&0.5312	&3.0876	&0.4786\\ 
	80\%	&3.7602	&0.6221	&0.5136	&3.7606	&0.4921\\
	70\%	&2.3195	&0.4429	&0.4058	&2.3214	&0.4152\\
	50\%	&3.5948	&0.3029	&0.3037	&3.5952	&0.4859\\
	30\%	&1.9598	&0.2247	&0.3402	&1.9615	&0.5322\\
	10\%	&0.8870	&0.2364	&0.2796	&0.8941	&0.3263\\
	 5\%	&1.0241	&0.0853	&0.3047	&1.0156	&0.3506
\end{tabular}
\end{table}

\begin{table} [htbp]
\centering
\caption{RMSE of (traffic, temporal, humidity)}
\label{table:traffic_temporal_hum}
\begin{tabular}{ r | r r r r r}
	testing	&LI	&TRMF	&Hybrid-kNN \\ \hline
	85\%	&16.6428	&5.3882	&5.3955\\ 
	80\%	&27.7509	&5.0636	&5.0810\\
	70\%	&21.4693	&5.5701	&5.0198\\
	50\%	&25.8277	&5.8469	&5.1926\\
	30\%	&27.4891	&5.0048	&5.0268\\
	10\%	&18.9358	&4.6402	&4.1944\\
	 5\%	&23.5126	&4.0820	&3.6661
\end{tabular}
\end{table}


\begin{table} [htbp]
\centering
\caption{RMSE of (traffic, temporal, temperture)}
\label{table:traffic_temporal_tem}
\begin{tabular}{ r | r r r r r}
	testing	&LI	&TRMF	&Hybrid-kNN \\ \hline
	85\%	& 8.6774	&2.0011	&1.8381\\ 
	80\%	&11.4353	&1.5281	&1.6787\\
	70\%	& 7.1857	&1.8447	&1.6644\\
	50\%	&10.5938	&1.7748	&1.5793\\
	30\%	&14.1343	&1.6737	&1.6874\\
	10\%	&16.1660	&1.4345	&1.4137\\
	 5\%	&10.0223	&1.4254	&1.3560
\end{tabular}
\end{table}

\subsubsection{Multivariate TRMF}
We see substantial improvement (if look at all to help) 

\begin{table}[htbp]
\caption{Multivariate TRMF on Traffic Data Set}
\label{traffic}
\begin{tabular}{r | r r r}
traffic dataset	&TRMF	&MtMF	&MtMF-all \\ \hline
humid 5-10-85	&4.6898	&4.6690	&3.0966\\
humid10-10-80	&3.5391	&3.5232	&2.3081\\
humid20-10-70	&2.5865	&2.5886	&1.8027\\
humid40-10-50	&1.9503	&1.9435	&1.5207\\
humid60-10-30	&1.6748	&1.6594	&1.4116\\
humid80-10-10	&1.5707	&1.5574	&1.3706\\
humid85-10- 5	&1.5098	&1.4963	&1.3301\\
 temp 5-10-85	&1.6235	&1.6105	&0.9620\\
 temp10-10-80	&1.2159	&1.2186	&0.7305\\
 temp20-10-70	&0.9017	&0.8936	&0.6002\\
 temp40-10-50	&0.6898	&0.6819	&0.5200\\
 temp60-10-30	&0.5865	&0.5818	&0.4770\\
 temp80-10-10	&0.5521	&0.5464	&0.4652\\
 temp85-10- 5	&0.5196	&0.5126	&0.4319\\
humid 5-10-85t	&5.3882	&5.3994	&5.1394\\
humid10-10-80t	&5.0636	&5.2424	&4.2964\\
humid20-10-70t	&5.5701	&5.6727	&4.2332\\
humid40-10-50t	&5.8469	&6.0046	&4.6036\\
humid60-10-30t	&5.0048	&4.9566	&3.7443\\
humid80-10-10t	&4.6402	&4.7294	&3.2574\\
humid85-10- 5t	&4.0820	&3.9843	&3.1709\\
 temp 5-10-85t	&2.0011	&2.0344	&1.5952\\
 temp10-10-80t	&1.5281	&1.7150	&1.2174\\
 temp20-10-70t	&1.8447	&1.9223	&1.1501\\
 temp40-10-50t	&1.7748	&1.7961	&1.0494\\
 temp60-10-30t	&1.6737	&1.6819	&1.2274\\
 temp80-10-10t	&1.4345	&1.4225	&0.9267\\
 temp85-10- 5t	&1.4254	&1.3374	&0.9259

\end{tabular}
\end{table}

\begin{table}[htbp]
\caption{Multivariate TRMF on Berkeley Data Set, random}
\label{traffic}
\begin{tabular}{r | r r r}
berkeley dataset	&TRMF	&MtMF	&MtMF-all \\ \hline
humid  5 &0.1936	&0.1941	&0.1594\\
humid 10 &0.1424	&0.1421	&0.1219\\
humid 20 &0.1140	&0.1139	&0.0975\\
humid 40 &0.0918	&0.0910	&0.0824\\
humid 60 &0.0815	&0.0808	&0.0762\\
humid 80 &0.0756	&0.0749	&0.0712\\
humid 85 &0.0750	&0.0740	&0.0713\\
 temp  5 &0.1136	&0.1158	&0.0694\\
 temp 10 &0.0454	&0.0482	&0.0371\\
 temp 20 &0.0316	&0.0329	&0.0264\\
 temp 40 &0.0232	&0.0233	&0.0201\\
 temp 60 &0.0181	&0.0183	&0.0166\\
 temp 80 &0.0155	&0.0155	&0.0144\\
 temp 85 &0.0158	&0.0159	&0.0142
\end{tabular}
\end{table}
\subsection{Heterogeneous Signal Incorporation}

We compare our TF method with the models without modeling heterogeneous sensor correlations in this section.
The Tensor Factorization naturally allow us to add additional nominal dimensions to the model, e.g.\ node id or sensor coordinates.
The three-order TF is used in the experiment, each dimension represent node id, time frame number and heterogeneous signal.  
Due to the indexes of each dimension are discrete, the heterogeneous sensor data are not able directly used.
Therefore, the discretization of heterogeneous data is needed.
Before training, the heterogeneous signal are divide into some bins according to the value.
Each bin represents the index of the dimension.
Table ? and ? show the TF result of random split and temporal split on Berkeley dataset while table ? and ? show the TF result of random split and temporal split on traffic dataset.

\subsection{Prediction Performance}
In this section, we measure the performance of missing value recovery algorithms by regression methods.
We demonstrate the experiment on traffic dataset.
The readings of gateway node are to be predicted in offline mode.
At the sane time the data from gateway is the label of a instance while the data from the other sensor nodes are the features of a instance.
80 percent of instance are split to be training data, the remaining part is the testing data.
Firstly, we filling the features from training and testing data by global mean, linear interpolation, KNN, and TRMF model.
The regression model we choose is linear regression(LR) and support vector regression(SVR).
The former is a linear model.
In contrast, the latter is a nonlinear model.
Table ? show the result with different data missing rate.
It is obvious that the more higher quality filling value will help the regression model predicting the objective sensor more precise.

\begin{table} [htbp]
\centering
\caption{predict gateway humidity by LR (RMSE) }
\label{table: LR}
   Filling method
\begin{tabular}{ r | r r r r r}
        missing rate&global mean     &LI   &Hybrid-KNN &TRMF\\ \hline
        5\%      &&&2.5121&\\
        10\%    &4.143&3.868& 2.597&2.526\\
        30\%    &5.161&3.955&2.773&2.470\\
        50\%    &6.234&4.282&3.158&2.756\\
        70\%   &7.728&5.082&3.310&2.743\\
        80\% &&&3.306&\\
\end{tabular}
\end{table}

\begin{table}[htbp]
\centering
\caption{predict gateway humidity by SVR (RMSE) }
\label{table: SVR}
   Filling method
\begin{tabular}{ r | r r r r r}
        missing rate&global mean     &LI   &Hybrid-KNN &TRMF\\ \hline
        5\%&&&12.1615&\\
        10\%    &3.933 &4.006&&2.591\\
        30\%    &5.172&4.083&&2.532\\
        50\%    &6.234&4.384&&2.813\\
        70\%   &7.686&5.235&&2.822
\end{tabular}
\end{table}
