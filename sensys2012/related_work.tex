\section{Related Work}\label{sec:rw}


%Missing data imputation is the process used to fill the missing data items by determine or assign values\cite{Little:hotdeck}. The physical characteristic of WSNs has shown that there are strong spatial and temporal correlation between sensor nodes and its past historical readings\cite{akyildiz2004exploiting}. 
%Sensors, according to their physical phenomenon, have multivariate correlations\cite{lou:multivariate_gap}. 
%\end{comment}

Table~\ref{tbl:methods} summarizes the data imputation methods in two dimensions: one describing the information utilized for imputation, and the other describing the imputation strategy. The missing-values filling strategy can be further categorized as hot-deck imputation and prediction models~\cite{Garcia:KNNreview}. 
Hot-deck imputation methods directly fill in the missing values using either neighbor values or historical records from itself 
such as the \textit{last-seen} method, while the prediction models exploit a function (involving metrics other than just sensor 
values) to estimate the missing values. 


%\begin{equation}
%y_{it} = f( y_{i(t-1)} , y_{j(t-1})
%Table~\ref{tbl:methods} summarize the sensory data imputation methods. The vertical dimension in Table~\ref{tbl:methods} shows whether it exploits the temporal, spatial, or hybrid hybrid of both information. The horizontal dimension classifies the methods by how the missing value is produced, either direct substitution or through prediction. 

\subsection{Temporal Imputation Methods}
Linear interpolation is a common temporal correlation based technique for missing data prediction. 
It is usually regarded as a baseline method, and has been implemented into some data analysis tools. Mathematically, the estimated value $\hat{y_{it}}$ for sensor $i$ at time $t$ is:
$\hat{y_{it}} = y_{iu} + \frac{y_{iv}-y_{iu}}{T_v-T_u}(t-T_u)$, where $y_{iu}$ ($y_{iv}$) at time $T_u$ ($T_v$) is the first 
observation for sensor $i$ prior to (after, respectively) time $t$.
Our Linear interpolation baseline does not considers correlation among sensors. In general, its performance is acceptable when the missing data are scattered. 
However, the quality of imputation goes down significantly when the sampling rate is low or facing consecutive data missing. 
Imputation based on the last-seen, moving average, and other statistics applied to the target sensor generally belong to the 
temporal-based methods as well.
%These methods share the advantage of comprehensible and easy implementation. 
%However,their performance depends significantly on the sensor data missing pattern as well as sampling rate.   

\subsection{Spatial Imputation Methods}
%The physical phenomenon of spatial correlation in sensors measurement can further be divided into two categories, the point source and the field source\cite{viran:spatialtemporal}. 
In some point source sensors where the strength of the sensed signal in a region satisfies certain physical laws, the imputation values can be obtained as the result of linear superposition of neighborhood sensors.
For instance, the Data Estimation using Physical Model (DEPM)~\cite{li2008data} method employs the basic laws of Physics and 
designs a prediction function as,  
%\begin{equation}
$I_K =\sum_{j=1}^M\frac{P_j}{4\pi d^2(I_j,s_k)}$
%\label{DEPM}
%\end{equation} 
where $I_k$ is the intensity of the target sensor value, $j$ represents the neighbor sensors, and P stands for the power radiated from one source to another. 
However, such models are only applicable to limited types of signals, and generally require the precise three-dimensional distance among sensors.

Researchers also propose to predict missing values based on data mining techniques. Window Association Rule Mining (WARM)~\cite{le2005estimating} and Freshness Association Rule Mining (FARM)~\cite{Gruenwald:FARM} studied the estimations of missing data based on the association rules among spatially-correlated neighbors. 
Such methods enjoy the advantage of being able to handle categorical sensor data, but the performance is limited in continuous sensor data due to the inherent limitations of association rules (basically, the continuous data needs to be quantized into suitable
categories for association matches).
%Moreover, the support, confidence of the mined rules need to be predefined by the users which can cause difficulties for building a satisfying model by users without profound knowledge about the environment of deployment. 
%\subsubsection{K-Nearest Neighbours Imputation} 
%  The two major drawbacks of KNN-based model that predicts only from the spatial correlation are: first of all,  they cannot capture temporal correlation, and the second reason is that most of these methods rely on the knowledge of distance between each sensor, which might not be available or accurate in the real world.

The Multi-Im (MI) method~\cite{yuan2000multiple} imputes the missing data by replacing each missing value with a set of plausible values instead of filling in a single value. Various methods including multiple linear regression, propensity score method, and Markov chain Monte Carlo method have been used in MI. The missing data are filled in $M$ times ($M=3-10$~\cite{Little:hotdeck}) 
to generate M complete datasets, and the results from M generated datasets are averaged to obtain a single prediction~\cite{yuan2000multiple}. 
This model suffers from high computation cost because multiple models have to be learned. Moreover, a poor model might unduly 
contribute to an inferior averaged result. 
%However, this method is criticized on various grounds by WSNs imputation researchers~\cite{jiang2007estimating,Gruenwald:FARM}. 
%First, in sensor data we do not know how many rounds of information to use in order to get the associated information.
%In addition, it is difficult to draw a pool of similar complete cases for a certain round of a certain sensor. 
%Last, it consumes unnecessary time since the sensor data may or may not related to all of the available information.



\subsection{Spatio-temporal Imputation Methods}

%Past studies indicate that imputation methods using only spatial or temporal information cannot produce satisfiable outcomes~\cite{Lim:robust}. 
%Given the weaknesses of prior imputation methods that use only temporal or spatial information,
%researchers have proposed various methods that exploit both temporal and spatial information.
%Below we highlight the salient such methods.

%In the environment when the radio transmission quality is poor or the packets collision happens frequently, exploiting only temporal information is not ideal. In contrast, the predictors using only spatial data can easily be harmed by the bias and drift from neighbourhood sensing data. Moreover, in some outdoor environment, the spatial information about the sensors could be erroneous or missing in themselves. Therefore, 

The Spatial and Temporal Imputation (STI) algorithm is proposed in~\cite{li2008spatial}. 
For each missing record, STI first checks if some nodes are within the sensing neighborhood, and utilizes
the average of the neighbors to impute the missing value. If no sensors are in the neighborhood, the last seen value of the 
missing sensor is used for imputation. The method favors spatial information over temporal information.

%It has the advantage of simple and fast, but suffers when last seen value is far away from the current time stamp, or when most of the nodes in the sensing region has high missing rate.
%Furthermore, the usage of circular region is problematic as most of the landforms might not be uniformly distributed. Finally, determining the range of the radius is tricky, as a larger radius can cause noise while a smaller one cannot highlight the effectiveness of this model(i.e. the results will be similar to linear interpolation).
%The radius of the region is sometimes too large for sensors whose correlations with the target sensor are low in the sensing range.  
Another spatio-temporal algorithm relying more on spatial than temporal is the Data Estimation using Statistical Model (DESM)~\cite{li2008data}.  
Given two sensors data streams, $\{X_{i1},X_{i2},\dots,X_{im}\}$, and $\{X_{j1},X_{j2},\dots,X_{jm}\}$, and assuming
the value of $X_{i(m+1)}$ is missing,
the prediction on sensor $i$ by its nearest sensor $j$ is:
\begin{center}
$\hat{X}_{i(m+1)} = (1-\alpha)X_{im} + (\alpha)\hat{z}$
\end{center}
where $\alpha$ is the Pearson correlation coefficient between the sensor $i$ and $j$, and $\hat{z}$ is the estimation of sensor $i$ 
based on the observation from sensor $j$---it measures the influence of the data sensed by node $i$ on the data of 
the nearest node $j$:
\begin{center}
$\hat{z} = X_{im}(1+\frac{X_{j(m+1)}-X_{j(m)}}{X_{j(m)}})$
\end{center}
%The above equation assumes that the sensor $X_i$ and $X_j$ have the similar data fluctuation trend. 
%$\alpha$ is expected to have more impact on the prediction if $X_i$ is more correlated with $X_j$. 
%\begin{equation}
%\label{z-hat}
%\end{equation}
%In DESM, because $\alpha$ is usually high around the target sensor. Therefore, the temporal feature multiplied by $1-\alpha$ holds a very small portion of contribution to the prediction process. It often contradicts the nature that the temporal correlation itself is usually higher than any spatial correlation with other sensors. The performance may becomes bad if the nearest sensor is not the most correlated one, i.e. blocked by a wall.
% \subsubsection{Applying K-nearest neighbour Estimation}

K-nearest neighbors algorithm (KNN) is an intuitive yet effective spatial-correlation-based imputation method. 
It directly uses the weighted average of the neighbor signals to impute the missing data, and has been adopted to successfully 
estimate the missing values of DNA micro-arrays~\cite{Troyanskaya:DNAKNN}.  
%While in WSNs, the sensor data of different nodes is more likely to have some functional relations other than just using sensor values.
Pan~\cite{pan2010k} proposes the Applying K-nearest neighbor Estimation (AKE) algorithm to exploit the spatial correlation in the 
missing sensory data problem. 
AKE adopts a linear regression model to describe the spatial correlation of nodes $i$ and $j$,
\begin{center}
$y_{it} = \alpha + \beta\cdot y_{jt} + \mu_{jt} $,
\end{center}
where $y_{it}$ is the observation of sensor $i$ at time $t$, $y_{jt}$ is the data of sensor $j$ at the same time, 
and $\mu_{jt}$ is the random error at time t.   
When the data of sensor $i$ at time $t$ is missing, the estimated value of sensor $i$ by sensor $j$ at the same time is:
\begin{center}
$\hat{y}_{it}^{(j)} =\hat{\alpha} +\hat{\beta}\cdot y_{jt}$
%\label{ake_j}
\end{center}
The $\hat{\alpha}$ and $\hat{\beta}$ are the estimated values of  $\alpha$ and $\beta$, based on the sample data according to the least squares principle.
%They claim that using a linear combination of the estimation from neighbour sensors can lower the random error caused by only using a single sensor. 
Therefore, given the $m$ nearest neighbor sensors who have values at time $t$[$y_{1t}, y_{2t},\dots, y_{mt}$], the missing value is predicted by :   
 \begin{center}
$ \hat{y}_{it} =\sum_{j=1}^m w_j \cdot \hat{y}_{it}^{(j)}$
 %\label{ake_impute}
 \end{center}
% First they build a matrix 
%\[ \left( \begin{array}{cccc}
%
%\end{array}
%\right)
%\]
where $w_j$ is the weight of sensor $j$ used to predict sensor $i$, and is larger for more highly correlated pairs.
They adopt the r-square statistics to rank their correlations. The value of $w_j$ is the weighted average of the corresponding 
r-square values for the $m$ sensors. 

The above models suffer a serious concern as they make some unverified assumptions about the data. For example, some models consider temporal correlation as more important than spatial correlation so nearby sensors' readings are used only when near-by time steps are missing (or vice versa), while some assume that sensors of the same distance should have similar correlation regardless of their 
orientation or other factors.  Such assumptions might or might not hold for various types of data, while imposing them a priori
to the model might lead to inconsistent results.  The models we propose, however, try not to rely so much on such a priori 
knowledge, and learn the latent correlation directly from the data.  The Imputation Method (ImM)~\cite{Lim:robust}, tries to learn
the appropriate weighting of temporal and spatial features by considering two temporal predictors, the last-seen predictor
and an autocorrelation-based temporal linear predictor, and one spatial linear predictor.

%Based on the (eq\ref{ake_j}), the $R^2$ is computed by:
%\begin{equation}
%R_(j)^2 = \sum_{n=1}^{h} \frac{(\hat{y}_{in}^(j)-\bar{y}_i)^2}{(y_{in}-\bar{y}_i)^2}
%\end{equation}
%Previous study [2] has showed that for high sampling rates sensing, the temporal correlation is higher than the spatial correlation from nearby sensors. Therefore, if there are less consecutive missing frames than the threshold learned by AKE , the linear interpolation is used to predict the value. However, in some environment, the distance used to decide the m nearest neighbours is not feasible. Moreover, in certain scenarios where two far away sensors have higher correlation, i.e. two sensors place under the same air-conditioning tuyere, the AKE will ignore the far away sensor. Finally, the preference on spatial over temporal is also being challenged when the sampling rates is very high, i.e. the body sensors, using temporal linear interpolation has better prediction.   

%\subsubsection{Imputation Method-ImM}
% In \cite{Lim:robust}, an imputation method(ImM) consists of two temporal and one spatial predictors was introduced.  The algorithm argues that it is not robust to set the preference for its spatial predictor over temporal predictor, or vice versa. They assume that the predictor which yields the most accurate results in the training dataset is used to predict the missing packets. 
%The first temporal predictor(LRS) uses the last transmitted packet as the predicted value eq\ref{lrs}.
%\begin{equation}
%\hat{y}_{t} =y_{t-1} \label{lrs}
%\end{equation} 
% The second temporal predictor uses a linear predictor(LP). Given $o$, the training samples of specific sensor $p$, and $O$ who is the training data of sensors other than $p$. The coefficients of the linear predictor is decided by autocorrelation method of autoregressive modelling. The predicted value by eq\ref{lp}:
%\begin{equation}
%\hat{y}_{t}^p = a(2)X_o(t-1) - a(3)X_o(t-2)- \dots -a(p+1)X_o(t-p) \label{lp}
%\end{equation} 
%where $p$ is the sensor node ID.  
%The spatial predictor uses the linear predictive model, given the nodes who are successively transmitted to the sink node, $p_1,p_2,\dots,p_L$, the predicted value is(eq\ref{ImM_Spatial}): 
%\begin{equation}
%\hat{y}_{t}^p = \sum_{l=1}^L a_lp_l
%\label{ImM_Spatial}
%\end{equation}
% where the weights is determined by weighted coefficients(WC) using least square method$[a_1,a_2,\dots,a_L]^T = (O^TO)^{-1}O^To$.
%However, the performance of the ImM may degrades if the changes in the continuous data stream are sharp in the high sampling rate dataset. 
%For example, the wind sensors in the windy climate area may affect the method applied on the target sensor.  

\subsection{Matrix Decomposition Models}

The Matrix Decomposition model is the one that is closest to our
solution, but applies a Singular Value Decomposition (SVD)-based
approach.  A salient example is the Empirical Orthogonal Functions
(EOF) model, which has been applied to oceanographic applications to
solve the problem of missing or unreliable satellite
data~\cite{beckers2003eof}.  EOF first fills in the missing values
(e.g., using all zeros or the mean values) and then performs SVD to
decompose the matrix. The first $k$ spatial-EOF (i.e., the top-$k$
singular values in SVD) is used to reconstruct the matrix (i.e.,
perform dimension reduction) for imputation, where $k$ is determined
through cross validation over the data.  Researchers have pointed out
several drawbacks in the SVD-based methods when applying to impute
large-scale sensors network data. First, given large data, performing
SVD is extremely computational demanding as the inverse of a dense
matrix is required. Second, it requires to fill all the missing values
before the decomposition can be performed, and imperfect initial
imputation (i.e., reseachers argue that filling all zeros is
questionable) can significantly hurt the final imputation results, in
particular when the missing ratio is
high~\cite{koren2009matrix,ke2005robust}.

%The major differences between those models and ours is that them rely more on the spatial correlation than temporal correlation, while our model does not require specific spatial infromation as it tries to learn sensor correlations from data. 
  
%\subsubsection{Matrix Factorization on Structure of Motion} 
%tructure from motion(SFM)\cite{tomasi1992shape} and motion estimation are among the most important application in computer vision.   A number of studies have investigated different approaches to the MF-based structure from motion problem. \cite{buchanan2005damped} performs the L2-norm minimization using the standard Newton algorithm with a damping factor. In \cite{okatani2011efficient}, they propose a new method incorporates the damping factor into the Wiberg method to solve the L2-norm problem. A L1-norm factorization using Wiberg method is proposed in\cite{eriksson2010efficient}. 
%They turn the objective function into a non-linear least square problem and adopt Gauss-Newton method to solve the problem. However, the size and ranks of the video data is different from the long-term and densely deployed sensor data. The rank of linear subspace in SFM has rank of three after it is subtracted by its column-wise mean. It is also noted that the video camera in essential is an univariate sensor device which is different from the nature of heterogeneity of sensor networks. 
