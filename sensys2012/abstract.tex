\begin{abstract}
Data sets gathered from sensor networks often suffer from a significant fraction of missing data, due to issues such as 
communication interference, sensor interference, power depletion, and hardware failure. 
Many standard data analysis tools such as classification engines, time-sequence pattern analysis modules, and even statistical tools are ill-equipped to deal will missing values---hence, there is a need to impute missing readings prior to analysis.

In this paper, we present novel imputation methods that take a ``Recommendation Systems'' view of the problem: 
the sensors and their readings at each time step are viewed as products and user product ratings, 
with the goal of estimating the missing ratings.
Sensor readings differ from product ratings, however, in that the former exhibit high correlation in both time and space.
To incorporate this property, we modify the widely successful Matrix Factorization (MF) approach for recommendation systems 
to model temporal correlations and learn latent relationships among sensors.
We evaluate the approach using two environmental sensor network datasets, one indoor and one outdoor, and
two imputation tasks, corresponding to intermittent readings and failed sensors.
The results show that our Temporally-regularized MF (TRMF) approach provides significantly higher estimation accuracy than 
both (i) state-of-the-art recommendation models and (ii) state-of-the-art sensor data imputation approaches 
such as the hybrid-KNN model.
Interestingly, adding spatial coordinate information into TRMF is shown to be ineffective---TRMF already
captures the latent relationships among sensors, including spatial correlations.

Next, we consider sensor networks with multiple sensor types at each node.  We present two techniques for extending 
TRMF to account for possible correlations among sensor types (e.g., temperature and humidity): Multivariate TRMF and 
Temporally-regularized Tensor Factorization.  Our results show that both techniques are significantly more accurate
than prior approaches, and each has its strengths, depending on the observed variance in the readings.
Finally, we consider a popular data analysis task---regression-based prediction---and show that, compared to prior approaches, 
applying our more-accurate imputation techniques leads to higher-quality prediction models.
\end{abstract}

%\category{H.4}{Information Systems Applications}{Miscellaneous}
%A category including the fourth, optional field follows...
%\category{D.2.8}{Software Engineering}{Metrics}[complexity measures, performance measures]

%\terms{Delphi theory}

\keywords{missing data recovery, matrix factorization, tensor factorization, temporal regularization, multivariate learning, data imputation}
